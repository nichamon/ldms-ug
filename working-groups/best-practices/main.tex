\documentclass{article}
\usepackage{graphicx} % Required for inserting images
\usepackage{hyperref}
\title{ldms-ug}
\author{Thomas Tucker}
\date{October 2023}

\begin{document}

\maketitle

\section{Introduction}
This document represents the community findings of best practices for:
\begin{itemize}
    \item Building LDMS from source
    \item Creating RPM
    \item Creating and deploying containers implementing LDMS system components
\end{itemize}.

\section{Build}
\subsubsection{Requirements}

\begin{itemize}
    \item Configure must discover and include development libraries where possible
    \item Features that have their dependencies met will automatically be built
    \item May be difficult with modules:
        \begin{itemize}
            \item User has to know which modules to load
            \item There are module version requirements
            \item Some modules are in conflict
        \end{itemize}
    \item ../configure must report missing dependencies and fail for features requested with –enable
    \item “make dist” tar file name should  encode major.minor.fix-<git-hash> in the file name
        \begin{itemize}
            \item Currently the major.minor.fix-``git-hash'' naming comes from the configure.ac
            \item The git hash labeling does not work in all cases
        \end{itemize}
    \item Configure should fail if the git hash cannot be obtained (e.g. building as root in a tree owned by ‘user’)
\end{itemize}

\subsection{Issues}
    \begin{itemize}
        \item There are a large number of configuration options Have the defaults (no options specified) build all features that are a common subset across supported linux distributions
        \item Multiple mechanisms are used to implement ‘optional’ features
            \begin{itemize}
                \item \_WITH
                \item \_ENABLE
                \item Env vars
                \item …
            \end{itemize}
        \item Many ‘optional’ features are not actually optional and will break the build if disabled, e.g. python
        \item Need to normalize the way optional features are implemented in configure
        \item Have a ‘what is enabled’ list of plugins/transports/auths dumped at the end of configure output. 
        \item Base OS cython on rhel8 is insufficient (ok on rhel9?). We may be able to influence toss cython default, but other OS’s not in our control.
        \item –disable-* does not work with the libraries that have all the dependencies.
            \begin{itemize}
                \item This may occur on a platform with a new compiler set the tree is not tested against.
                \item Autoconf may estimate a compatible dependency library is available when in fact it isn’t.
                \item Build target may lack an external dependency present on the build host. (building extras then forces manual deletion in a packaging or deployment process).
            \end{itemize}
    \end{itemize}

\subsection{Status}
    \begin{itemize}
        \item All work items are currently unassigned to developers
        \item Need to provide a repository to publish user-group content under github
    \end{itemize}
    
\section{RPM}
\begin{itemize}
    \item Assumptions
        \begin{itemize}
                \item Python3 and python3-devel are required to build all RPM
                \begin{itemize}
                    \item python3 support is presumed to be present on all target systems
                \end{itemize}
                \item Any of the ‘standard’ packages that don’t build will break the RPM build process
        \end{itemize}
   \item RPM version must be consistent with the version reported by tools
        \begin{itemize}
                \item ldms\_ls -V
                \item ldmsd -V
                \item Should be able to identify the RPM and/or local build that is currently running
                \item configure must be run to update the CPP macros that are used by software to report version if you don’t re-run configure, you’ll get “old” information
                \item should implement procedure to appropriately create tags where necessary
        \end{itemize}        
    \item Naming convention
        \begin{itemize}    
            \item There are lots of docs on this sort of thing that we should comply with
                \begin{itemize}
                    \item \href{http://ftp.rpm.org/max-rpm/ch-rpm-file-format.html}{http://ftp.rpm.org/max-rpm/ch-rpm-file-format.html}
                    \item \href{https://docs.fedoraproject.org/en-US/packaging-guidelines/Versioning/}{https://docs.fedoraproject.org/en-US/packaging-guidelines/Versioning/}
            \end{itemize}
        \item If git hash matches release level:
            \begin{itemize}
                \item Major.minor.fix
            \end{itemize}
        \item otherwise
            \begin{itemize}
                    \item major.minor.fix-<patchcount>-<short-hash>
                    \item Rpm packagers often have to create a local tag with somehow meaningful version name (if they want to avoid seeing patch-hash suffix.
                    \item Currently the short-hash is not computed correctly in all cases.
            \end{itemize}
       \end{itemize}
    \item make dist tar file output naming must match ldms\_ls -V output
    \item RPM package for all commands and tools and dependent libraries
    \item RPM package for ‘standard samplers and stores’
        \begin{itemize}
            \item Any sampler that will build regardless of dependencies
                \begin{itemize}
                    \item meminfo
                    \item vmstat
                    \item procstat
                    \item procnetdev
                    \item ??
                \end{itemize}
            \item Any store sampler that will build without dependencies
               \begin{itemize}
                    \item store
                    \item json
                    \item flatfile
                    \item store\_none
                    \item store\_function
                    \item ??
                \end{itemize}
        \end{itemize}
    \item RPM for transports that are “always” available
        \begin{itemize}
            \item socket
        \end{itemize}
    \item RPM package for each of the special purpose samplers that require dependencies
        \begin{itemize}
            \item IB package
            \item Lustre package
            \item Slingshot package
            \item slurm package
            \item Aries package
            \item Cray package
            \item Nvidia package
        \end{itemize}
    \item RPM Package for each store (each store has its own set of independent dependencies). tt: Many of these storage plugins will/should be deprecated as part of moving the core technology stores to support decomposition
        \begin{itemize}
            \item avro\_kafka
            \item darshan
            \item influx
            \item json
            \item kafka
            \item kokkos
            \item kokkos\_appmon
            \item papi
            \item proc\_store
            \item rabbit
            \item slurm
            \item store\_amqp.c
            \item store\_app
            \item store\_rabbitkw.c
            \item store\_rabbitv3.c
            \item store\_sos.c
            \item stream
            \item stream\_dump
        \end{itemize}
    \item Packagers please add your suggestions!
    \item SNL package split:
        \begin{itemize}
            \item main(all binaries that got built using/requiring the ldmsd binary)
            \item Python layer (maybe subsumed in main)
            \item Doc layer (man, html, etc)
            \item Any local value-added plugin rpm(s).
            \item Daemon configuration rpm (genders or maestro or ansible or whatever)
            \item All the above could be controlled by rpm ‘with’ options and merged into a single rpm output at the whim of the system owner who builds their own deviant rpms.
        \end{itemize}
\end{itemize}

\section{Containers}
\subsection{Build Containers}
These containers include all of the tools and dependencies necessary to build the top of tree on a white-box Linux platform. These will work on Cray platforms, but for reasons of proprietary library licensing (e.g. Slingshot), they may not be able to build transport plugins for proprietary networks.

The build containers are useful for building other containers as described below
\subsection{Generic Containers (“Runtime” containers?)}
These containers provide a platform for deploying LDMS Aggregators and Samplers but with your own configuration. They contain pre-built Maestro and ldmsd along with transport, sampler, authentication and storage plugins.

\subsubsection{Transport Plugins}
\begin{itemize}
    \item socket
\item infiniband
\item libfabric
\end{itemize}
\subsubsection{Authentication Plugins}
\begin{itemize}
    \item none
    \item ovis (we need to rename this or create some reasonable alias such as ‘shared-secret’)
    \item munge
\end{itemize}
\subsubsection{Sampler Plugins}
\begin{itemize}
\item meminfo
\item vmstat
\item procstatutil
\item …
\end{itemize}

\subsection{Storage Plugins}
\begin{itemize}
\item SOS
\item CSV
\item Avro\_Kafka
\item . . .
\end{itemize}

\section{Demo/Pre-configured Containers}
These containers are derived from the Generic Containers.

These are pre-configured deployments of a Sampler, and an Aggregator. The transport configuration will be sockets. The intention is that a user could trivially stand up a monitoring infrastructure with very little effort. These containers are also a useful starting point for creating a customized deployment, for example over Infiniband instead of sockets.

Reasonable defaults are included for all tunables, for example sampler interval. These can be overridden when the container is started with arguments to the docker run command. For example:

\begin{verbatim}
    docker run sampler i_am=001 interval=10s
\end{verbatim}

Host naming will be normalized and controlled by parameters passed to the docker run command, e.g.:

\begin{verbatim}
    docker run aggregator \
        i_am=agg-01 \
        samplers=sampler-[001-100] \
        storage=/shared/agg-01/ldmsd
\end{verbatim}

\subsection{Status}
The following containers are currently available and maintained on docker-hub:
\begin{itemize}
\item https://hub.docker.com/r/ovishpc/ldms-dev
\item https://hub.docker.com/r/ovishpc/ldms-samp
\item https://hub.docker.com/r/ovishpc/ldms-agg
\item https://hub.docker.com/r/ovishpc/ldms-storage
\item https://hub.docker.com/r/ovishpc/ldms-web-svc
\item https://hub.docker.com/r/ovishpc/ldms-grafana
\item https://hub.docker.com/r/ovishpc/ldmscon2023
\end{itemize}
All of the containers are based on ubuntu:22.04.

The github repository for the source code for building the containers can be found at https://github.com/ovis-hpc/ldms-containers

\end{document}

